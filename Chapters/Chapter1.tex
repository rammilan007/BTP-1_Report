% Chapter 1

\chapter{Introduction} % Main chapter title

\label{Chapter1} % For referencing the chapter elsewhere, use \ref{Chapter1} 

%----------------------------------------------------------------------------------------

% Define some commands to keep the formatting separated from the content 
\newcommand{\keyword}[1]{\textbf{#1}}
\newcommand{\tabhead}[1]{\textbf{#1}}
\newcommand{\code}[1]{\texttt{#1}}
\newcommand{\file}[1]{\texttt{\bfseries#1}}
\newcommand{\option}[1]{\texttt{\itshape#1}}

%----------------------------------------------------------------------------------------CHAPTER 1----------------------------------------------------

\section{Motivation}
The motivation for taking this project lies in its huge potential and application in the field of data science.

Also talk about popularity and future of this field.\\
IMUs are popular component of Inertial Navigation system. But these sensors suffer from drift with time. And hence they can not be used without aiding with another sensors for a longer duration of time.

%----------------------------------------------------------------------------------------

\section{Objective}

The main objective is state(attitude) estimation. The application of this estimation techinques to estimate the various parameters of UAVs equipped with Visual and inertial sesnsors.

This requires first understanding the system, modelling the physics taking into account the various primary effects and then simulation is required to get the idea of bounds or errors or how well the estimator is working with our system.
\section{Structure of the Report}

\keyword{Chapters} -- The report has been structured into  five chapters as follows:
\begin{itemize}
\item Chapter 1: Introduction and objective \ref{Chapter1}
\item Chapter 2: Literature Survey
\item Chapter 3: Various State Estimation Techniques and their validation
\item Chapter 4: Modeling and Simulation
\item Chapter 5: Conclusions and future directions
\item Chapter 6: Bibliography
\end{itemize}

%----------------------------------------------------------------------------------------CHAPTER 2----------------------------------------------------
\chapter{Literature Survey}
\label{chapter 2}

\subsection{Various problems/difficulties encountered by researchers}

\begin{itemize}
    \item Time delays of in sensors
    \item Time offset between various sensors
    \item different sampling rate for different sensors
    \item need of online calibration
    \item camera suffers form imaage blur and global shutter and rolling shutter effects.
    \item Known map based - conditions under which images was taken 
    
\end{itemize}

\subsection{Approaches towards those problems}
Talk about the various approaches used by researchers to tackle those problems. For varying time delays this paper included the delay time between camera and IMU in the state vector of the state dynamics of the system and tried to estimate online.
%----------------------------------------------------------------------------------------CHAPTER 3----------------------------------------------------
\chapter{State Estimation}
\label{chpater 3}
Talk about state estimation . Filtering, prediction and smoothing. Explain them. Also talk about weighted least squares problem, batch estimation and their suitability of online implimentation
\section{Kalman Filter}
Kalman filter is the most popular ways of online estimation techiniques

\section{Extended Kalman Filter}
Since most of the real physical systems can be modelled as nonliner differential equations, and Kalman filter can not be applied to these systems. Hence EKF is the extension of the Kalman to Filter for nonlinear systems.
\section{Unscented Kalman Filter}
This is deterministic sampling based filter. This has its own advantages. 
This assumes a symmetric distribution of random variables. 
\section{Validition of Filters}
The estimated states can be compared with the true states.
\subsection{RMSE}
Root Mean Square Error
\subsection{NESS}
chi distribution , number of degree of freedom as number of states.
%----------------------------------------------------------------------------------------CHAPTER 4----------------------------------------------------
\chapter{Modelling and Simulation}
saying ' no model is perfect'
modelling is an essential part. Estimation is highly dependent of model. The most accurate models will produce the most accurate estimates i.e., more closer the true value with less uncertainties.

This chapter also contains results and graphs of various simlulations.
\section{Sensor Modelling and True state generation}
how to generate true states. Assumption about Gaussian, Central limit theorem
\section{System Modeling}
\subsection{System 1:}
Four Tank dynamics system

\subsection{System 2:}
Induction Machine

\section{Simulation Results}
put all the graphs for all all the systems under consideration

%----------------------------------------------------------------------------------------CHAPTER 5----------------------------------------------------
\chapter{Conclusions and future directions}
Talk about estimation , effect of linearisation , and all, performance of filters , bullted future work



%----------------------------------------------------------------------------------------

\subsection{References}

The \code{biblatex} package is used to format the bibliography and inserts references such as this one \parencite{Reference1}. The options used in the \file{main.tex} file mean that the in-text citations of references are formatted with the author(s) listed with the date of the publication. Multiple references are separated by semicolons (e.g. \parencite{Reference2, Reference1}) and references with more than three authors only show the first author with \emph{et al.} indicating there are more authors (e.g. \parencite{Reference3}). This is done automatically for you. To see how you use references, have a look at the \file{Chapter1.tex} source file. Many reference managers allow you to simply drag the reference into the document as you type.

Scientific references should come \emph{before} the punctuation mark if there is one (such as a comma or period). The same goes for footnotes\footnote{Such as this footnote, here down at the bottom of the page.}. You can change this but the most important thing is to keep the convention consistent throughout the thesis. Footnotes themselves should be full, descriptive sentences (beginning with a capital letter and ending with a full stop). The APA6 states: \enquote{Footnote numbers should be superscripted, [...], following any punctuation mark except a dash.} The Chicago manual of style states: \enquote{A note number should be placed at the end of a sentence or clause. The number follows any punctuation mark except the dash, which it precedes. It follows a closing parenthesis.}

The bibliography is typeset with references listed in alphabetical order by the first author's last name. This is similar to the APA referencing style. To see how \LaTeX{} typesets the bibliography, have a look at the very end of this document (or just click on the reference number links in in-text citations).

\subsubsection{A Note on bibtex}

The bibtex backend used in the template by default does not correctly handle unicode character encoding (i.e. "international" characters). You may see a warning about this in the compilation log and, if your references contain unicode characters, they may not show up correctly or at all. The solution to this is to use the biber backend instead of the outdated bibtex backend. This is done by finding this in \file{main.tex}: \option{backend=bibtex} and changing it to \option{backend=biber}. You will then need to delete all auxiliary BibTeX files and navigate to the template directory in your terminal (command prompt). Once there, simply type \code{biber main} and biber will compile your bibliography. You can then compile \file{main.tex} as normal and your bibliography will be updated. An alternative is to set up your LaTeX editor to compile with biber instead of bibtex, see \href{http://tex.stackexchange.com/questions/154751/biblatex-with-biber-configuring-my-editor-to-avoid-undefined-citations/}{here} for how to do this for various editors.

\subsection{Tables}

Tables are an important way of displaying your results, below is an example table which was generated with this code:

{\small
\begin{verbatim}
\begin{table}
\caption{The effects of treatments X and Y on the four groups studied.}
\label{tab:treatments}
\centering
\begin{tabular}{l l l}
\toprule
\tabhead{Groups} & \tabhead{Treatment X} & \tabhead{Treatment Y} \\
\midrule
1 & 0.2 & 0.8\\
2 & 0.17 & 0.7\\
3 & 0.24 & 0.75\\
4 & 0.68 & 0.3\\
\bottomrule\\
\end{tabular}
\end{table}
\end{verbatim}
}

\begin{table}
\caption{The effects of treatments X and Y on the four groups studied.}
\label{tab:treatments}
\centering
\begin{tabular}{l l l}
\toprule
\tabhead{Groups} & \tabhead{Treatment X} & \tabhead{Treatment Y} \\
\midrule
1 & 0.2 & 0.8\\
2 & 0.17 & 0.7\\
3 & 0.24 & 0.75\\
4 & 0.68 & 0.3\\
\bottomrule\\
\end{tabular}
\end{table}

You can reference tables with \verb|\ref{<label>}| where the label is defined within the table environment. See \file{Chapter1.tex} for an example of the label and citation (e.g. Table~\ref{tab:treatments}).

\subsection{Figures}

There will hopefully be many figures in your thesis (that should be placed in the \emph{Figures} folder). The way to insert figures into your thesis is to use a code template like this:
\begin{verbatim}
\begin{figure}
\centering
\includegraphics{Figures/Electron}
\decoRule
\caption[An Electron]{An electron (artist's impression).}
\label{fig:Electron}
\end{figure}
\end{verbatim}
Also look in the source file. Putting this code into the source file produces the picture of the electron that you can see in the figure below.

\begin{figure}[th]
\centering
\includegraphics{Figures/Electron}
\decoRule
\caption[An Electron]{An electron (artist's impression).}
\label{fig:Electron}
\end{figure}

Sometimes figures don't always appear where you write them in the source. The placement depends on how much space there is on the page for the figure. Sometimes there is not enough room to fit a figure directly where it should go (in relation to the text) and so \LaTeX{} puts it at the top of the next page. Positioning figures is the job of \LaTeX{} and so you should only worry about making them look good!

Figures usually should have captions just in case you need to refer to them (such as in Figure~\ref{fig:Electron}). The \verb|\caption| command contains two parts, the first part, inside the square brackets is the title that will appear in the \emph{List of Figures}, and so should be short. The second part in the curly brackets should contain the longer and more descriptive caption text.

The \verb|\decoRule| command is optional and simply puts an aesthetic horizontal line below the image. If you do this for one image, do it for all of them.

\LaTeX{} is capable of using images in pdf, jpg and png format.

\subsection{Typesetting mathematics}

If your thesis is going to contain heavy mathematical content, be sure that \LaTeX{} will make it look beautiful, even though it won't be able to solve the equations for you.

The \enquote{Not So Short Introduction to \LaTeX} (available on \href{http://www.ctan.org/tex-archive/info/lshort/english/lshort.pdf}{CTAN}) should tell you everything you need to know for most cases of typesetting mathematics. If you need more information, a much more thorough mathematical guide is available from the AMS called, \enquote{A Short Math Guide to \LaTeX} and can be downloaded from:
\url{ftp://ftp.ams.org/pub/tex/doc/amsmath/short-math-guide.pdf}

There are many different \LaTeX{} symbols to remember, luckily you can find the most common symbols in \href{http://ctan.org/pkg/comprehensive}{The Comprehensive \LaTeX~Symbol List}.

You can write an equation, which is automatically given an equation number by \LaTeX{} like this:
\begin{verbatim}
\begin{equation}
E = mc^{2}
\label{eqn:Einstein}
\end{equation}
\end{verbatim}

This will produce Einstein's famous energy-matter equivalence equation:
\begin{equation}
E = mc^{2}
\label{eqn:Einstein}
\end{equation}

All equations you write (which are not in the middle of paragraph text) are automatically given equation numbers by \LaTeX{}. If you don't want a particular equation numbered, use the unnumbered form:
\begin{verbatim}
\[ a^{2}=4 \]
\end{verbatim}


